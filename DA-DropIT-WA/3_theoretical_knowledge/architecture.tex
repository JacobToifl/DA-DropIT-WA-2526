\section{Architekturen für Dokumentensysteme}\label{chapter:Architekturen_fuer_Dokumentensysteme}
\subsection{Suche und Filter-Architekturen}
\subsection{Rollen- und Berechtigungssysteme}
\newpage
\subsection{Skalierbare plattformunabhängige Systemarchitekturen}

\subsubsection{Skalierbarkeit}
Unter einer skalierbaren Systemarchitektur versteht man ein System, das sich problemlos an steigende Anforderungen anpassen lässt. Fehlende Skalierbarkeit führt 
dazu, dass Systeme bei wachsender Nutzerzahl oder zunehmendem Datenvolumen spürbare Performanceprobleme entwickeln.
Während des Entwicklungsprozess sind unterschiedliche Prinzipien zu beachten:

\begin{table}[h!]
\centering
\caption{Merkmale von Softwaresystemen}   
\label{tab:merkmale}                      
\begin{tabular}{|p{4cm}|p{10cm}|}
\hline
\textbf{Merkmal} & \textbf{Beschreibung} \\ \hline

Modularität &
Einzelne Komponenten der Software sollen unabhängig sein und sich flexibel aktualisieren oder erweitern lassen. \\
\hline

Flexibilität &
Systeme müssen sich dynamisch an verändernde Anforderungen anpassen können. \\
\hline

Fehlertoleranz &
Das System soll auf Fehler reagieren können und automatisch geeignete Maßnahmen ergreifen. \\
\hline

\end{tabular}
\end{table}

Um eine Skalierbare Architektur zu gewährleistem, sind Optimierungsfunktionen von Bedeutung, um die Effizienz und Leistung zu optimieren. 
Wichtige Optimierungsstrategien sidn unteranderm: 

\begin{table}[h!]
\centering
\caption{Techniken zur Leistungsoptimierung}
\label{tab:optimierung}
\begin{tabular}{|p{4cm}|p{10cm}|}
\hline
\textbf{Technik} & \textbf{Beschreibung} \\ \hline

Load Balancing &
Verteilung von Anfragen auf mehrere Server, um Überlastung zu vermeiden. \\
\hline

Caching &
Speicherung häufig abgerufener Daten, um Zugriffszeiten zu verkürzen. \\
\hline

Partitioning &
Aufteilung von Daten in kleinere Einheiten, die parallel verarbeitet werden können. \\
\hline

Asynchrone Verarbeitung &
Ermöglicht es Systemen, Aufgaben im Hintergrund auszuführen, was die Reaktionszeit verbessert. \\
\hline

\end{tabular}
\end{table}

Das Hautziel einer Skalierbare Architektur ist es , eine hohe Verfügbarkeit und Leistung sicherzustellen, auch bei steigender Nutzerzahl und wachsendem Datenvolumen.
Zudem soll die Wartbarkeit und Erweiterbarkeit des Systems gewährleistet werden, um zukünftige Anforderungen problemlos integrieren zu können.
Abschließend ist zu sagen, dass eine gut durchdachte skalierbare Architektur entscheidend für den langfristigen Erfolg von Softwaresystemen ist.

\newpage

\subsubsection{Platformunabhängigkeit}

Ein Computerprogramm benötigt eine Umgebung, in der es gestartet werden kann und während der gesamten Laufzeit stabil funktioniert.
Ein Programm gilt als plattformunabhängig oder plattformübergreifend, wenn es auf verschiedenen Computersystemen ausgeführt
werden kann – also auf Geräten mit unterschiedlicher Hardware, verschiedenen Prozessoren oder unterschiedlichen Betriebssystemen.
Der Grad dieser Unabhängigkeit wird als Portierbarkeit (oder Portabilität) bezeichnet. 

Die Portabilität kann z.\,B. geschätzt werden über
\[
  P = 1 - \frac{U + A}{E}
\]
mit
\begin{itemize}
  \item Übertragungsaufwand $U$ (insbesondere Neukompilierung),
  \item Anpassungsaufwand $A$ (Änderung des Quellcodes, z.\,B. bei Austausch von
        Betriebsschnittstellen),
  \item Entwicklungsaufwand $E$ für Neuentwicklung.
\end{itemize}

Dabei entspricht
\begin{itemize}
  \item eine Portabilität von $1$ der Kompatibilität, das Programm ist also ohne
        Änderung auf dem Zielsystem lauffähig:
        \[
          P = 1 \iff U + A = 0
        \]
  \item eine Quellcode-Portabilität in der Regel einer Gesamt-Portabilität von
        $> 90\%$: 
        \[
          A = 0 \Rightarrow P = 1 - \frac{U}{E} > 0{,}9 \iff U < 0{,}1E
        \]
  \item eine Portabilität nahe $0$ einer naheliegenden Neuentwicklung des
        Programmes:
        \[
          P \approx 0 \Rightarrow U + A \approx E
        \]
\end{itemize}

Damit ist nicht nur gemeint, dass 
ein Programm auf mehreren Plattformen laufen kann, sondern auch, wie viel Aufwand nötig ist, um es dafür anzupassen.
Dieser Vorgang wird Portierung oder Migration genannt.




\subsection{Ableitung zur Forschungsfrage von Jacob Toifl}